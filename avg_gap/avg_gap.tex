\documentclass[11pt]{article}

\usepackage[cp1251]{inputenc}
\usepackage[english]{babel}
\usepackage{amsmath}
\usepackage{amsthm}
\usepackage{amsfonts}
\usepackage{amssymb}
\usepackage{verbatim}
\usepackage[T1]{fontenc}
\usepackage{hyperref}
\usepackage{enumerate}


\DeclareMathOperator{\supp}{supp}
\DeclareMathOperator{\roots}{roots}
\DeclareMathOperator{\AG}{AG}
\DeclareMathOperator{\G}{G}
\DeclareMathOperator{\avg}{avg}

\newtheorem{theorem}{Theorem}[section]
\newtheorem{lemma}[theorem]{Lemma}
\newtheorem{proposition}[theorem]{Proposition}
\newtheorem{corollary}[theorem]{Corollary}
\newtheorem{definition}[theorem]{Definition}

\newcommand{\R}{\mathbb{R}}
\newcommand{\e}{\varepsilon}


\begin{document}
\begin{center}
\textbf{The sizes of the gaps between roots after applying $(1-D)$ operator}\\
Stanislav Atanasov and Christopher Shriver
\end{center}

\section{Motivation}
Real-rooted polynomials (also referred to as \emph{hyperbolic polynomials}) are these polynomials who have only real roots. They have been object of many recent research~\cite{Borcea1, Borcea2, Spielman, Fisk} and their theory has had a significant progress in recent years~\cite{Borcea1, Borcea2}. A vast body of research deals with analyzing the linear transformation that map real-rooted polynomials. We say that a linear transformation $T:\mathbb{R}[x]\to\mathbb{R}[x]$ \emph{preserves real-rootedness}, or \emph{preserve hyperbolicity} if $T(f)$ is real-rooted, whenever $f$ is real-rooted. A seminal result by Borcea and Br$\ddot{a}$nden is the complete characterization of all linear transformations that preserve stability of multivariate polynomials, which in the univariate case amounts precisely to characterization of all linear real-rootedness preservers.

Despite its generality, the result by Borcea and Branden also has its limitations. It only asserts that a particular linear transformation $T:\mathbb{R}[x]\to\mathbb{R}[x]$ preserves real-rootedness but does not shed any light on where on the real line do the roots of $T(f)$ lie as some dependance on the roots of $f$. Instead of analyzing the position of the roots of $T(f)$, we investigate a related question. We observe the gaps between these roots after repeated application of a particular linear transformation which preserves real-rootedness.

In this paper we focus our attention to a particular linear operator that appears often in the recent papers by Marcus, Spielman and Srivastava~\cite{Spielman}. The linear operator $(1-D)$, where $D=\frac{d}{dx}$ is the differential operator, is known to preserve real-rootedness. Throughout the paper, we use $\lambda_{min}=\lambda_1\leq \cdots \leq \lambda_n = \lambda_{max}$ to denote the roots of the real rooted polynomial $P(x)$. Of particular interest for us are the \emph{average gap}, $\AG(P)$, of the polynomial $P(x)$, defined by $\AG(P)=\frac{\lambda_{max} - \lambda_{min}}{\deg P - 1}$, as well as the gaps between consecutive roots, which we will denote by $d_{i+1}(f) := \lambda_{i+1}(f) - \lambda_i(f)$. We establish the following two results

\begin{theorem}
please fill the exact formula
\end{theorem}

\begin{theorem}
Suppose there is a sequence $\{f_k\}_1^\infty$ of real-rooted polynomials defined by $f_k(x) = (1-D)^k f_0(x)$, where $f_0(x)$ is real-rooted of degree $m$. Then
\[\lim_{n\to\infty} \min_{i \in [m]} d_i (f_n) = \infty\]
holds true.
\end{theorem}
\section{Gaps}

\subsection{Average gap after many iterations}
 In this section, we show that the average gap approaches infinity when we apply $(1-D)$ operator many times. More precisely, for any sequence of real-rooted polynomials $\{f_k\}_1^{\infty}$, satisfying $f_{k+1}(x) = (1-D)^k f_{1}(x),$ it holds that $\lim_{n\to \infty} \AG(f_n) = \infty$. We prove this claim in two short steps.

 \begin{lemma}
 \label{degree two avg distance}
 Let $f(x)$ be a real-rooted quadratic polynomial with roots $\lambda_1 \leq \lambda_2$. Then $\AG((1-D)^n f) \to \infty$ as ${n\to \infty}$.
 \end{lemma}
\proof
Without loss of generality, we might assume that $f(x)$ is monic and write it as $x^2+ax+b$. The Vieta's formulas yield to $\AG(f) = \lambda_2 - \lambda_1 = \sqrt{a^2 - 4b}$. Introduce the sequence $\{f_k\}_{0}^\infty$ of polynomials given by $f_{k} (x)= (1-D)^k f_0(x)$ and $f_0(x) = f(x)$. Since $(1-D)$ preserves real-rootedness all $f_k(x)$ are real-rooted. In similar fashion, denote by $\lambda_1^{(k)} \leq \lambda_2^{(k)}$ the two real roots of $f_k(x)$. Using Vieta's formulas once again, we obtain
 \[\AG(f_k) = \sqrt{\AG(f_{k+1})^2 + 4}.~(*)\]

 Clearly, $\AG(f_k) > \AG(f_{k-1})$ for every $k\ge 1$ and so  $\lim \AG(f_k)$ exists (possibly, and as we shall see, infinite). Assume it is finite, say $l$. Taking the limit on both sides of $(*)$ leads to $l = \sqrt{l^2 + 4},$ which is evidently impossible. Hence, $\lim \AG(f_k) = \infty$. \qed

\begin{lemma}
\label{degree two - slowest movement}
Let $f(x)$ be a real-rooted polynomial with roots $\lambda_{min}=\lambda_1 < \lambda_2 \cdots \leq \lambda_{n-1} < \lambda_n = \lambda_{max}$. Then $AG((1-D)f) \geq AG[(1-D)f_{1,n}]$, where $f_{1,n} = (x - \lambda_{min})(x - \lambda_{max})$ is the quadratic polynomial with roots $\lambda_1$ and $\lambda_n$.
\end{lemma}
\proof Note that since $\lambda_{min}$ and $\lambda_{max}$ are simple roots, they are not roots of $(1-D)f$. Thus, the biggest and smallest root of $(1-D)f$ are the biggest and smallest root of $\frac{(1-D)f}{f}$. Let $x_0\in\mathbb{R}$ be biggest root of $(1-D)f(x)$. By the above remark, we have
\[\frac{(1-D)f}{f}(x_0) = 0 \Leftrightarrow \sum_{i=1}^n \frac{1}{x_0 - \lambda_i} = 1.\]

Note that $x_0 > \lambda_i,~\forall 1\leq i \leq n$ and so $\frac{1}{x_0 - \lambda_1} + \frac{1}{x_0 - \lambda_n} < \sum_{i=1}^n \frac{1}{x_0 - \lambda_i} = 1.$ In particular, since $\frac{(1-D)f_{1,n}}{f_{1,n}}(x) = \frac{1}{x - \lambda_1} + \frac{1}{x - \lambda_n}$ is monotonically decreasing function on $(\lambda_n, +\infty)$, the above inequality implies that the biggest root of $(1-D)f_{1,n}$ is smaller than $x_0$. Similarly, we prove that the smallest root of $(1-D)f_{1,n}$ is bigger than the smallest root of $(1-D)f$. Hence, the claim follows. \qed

Now, combining the two lemmata, we easily derive

\begin{proposition}For any sequence of real-rooted polynomials $\{f_k\}_0^{\infty}$ satisfying $f_{k+1}(x) = (1-D)^k f_{0}(x),$ we have $\lim_{n\to \infty} \AG(f_n) =  \infty.$
\end{proposition}
\label{avg gap goes to infty}
\proof Note that if $\lambda$ is a root of multiplicity $k$, then it is a root of multiplicity $k-1$ of $(1-D)f$. Therefore, after $deg f$ applications of the $(1-D)$ operator, we obtain a polynomial with simple roots. Therefore, we might, without loss of generality, assume that $f_0$ has simple roots. Otherwise, pick $f_{deg f_0}$ as a first element of the sequence. Suppose $\lambda_{min}=\lambda_1 < \lambda_2 < \cdots < \lambda_{n-1} < \lambda_n = \lambda_{max}$ are the roots of $f_0(x)$. Now repeated application of Lemma~\ref{degree two - slowest movement}, coupled with the observation that $\AG((1-D)[(x-a)(x-b)]) \geq \AG((1-D)[(x-c)(x-d)])$ whenever $c < a,~b<d$, leads to

\[\AG(f_k)= \AG((1-D)^kf_0)\geq\AG((1-D)^k [(x-\lambda_{min})(x-\lambda_{max})]),\]
for every $k\in\mathbb{N}$. Lemma~\ref{degree two avg distance} shows that, as $k$ grows, the right hand side goes to $\infty$, and thus so does the expression on the left. \qed

\subsection{The growth of $i$-th partial gap}
In the previous section, we concluded that the average gap increases without bound upon repeated applications of the $(1-D)$ operator. Proposition~\ref{avg gap goes to infty} asserts that the average gap goes to infinity. A natural question is to ask whether the same thing holds for each gap between two consecutive roots. As we shall see, this is indeed true.

\begin{lemma}
\label{minimal gap increases}
 Consider sequence of real-rooted polynomials $\{f_k\}_1^\infty$ given by $f_k(x) = (1-D)^k f_0(x)$. Then
 \[\min |\lambda^{(n+1)}_{i+1} - \lambda^{(n+1)}_i | \geq \min |\lambda^{(n)}_{i+1} - \lambda^{(n)}_i |\]
\end{lemma}
\proof Chris, please add it.
\qed

\begin{comment}
\begin{lemma}
Suppose $f_0(x)$ is a real-rooted polynomial with distinct roots $\lambda_1 < \lambda_2 < \cdots <\lambda_{n}$ such that $\lim_{m\to\infty} |\lambda_{i+1}[(1-D)^m f_0] - \lambda_{i}[(1-D)^m f_0]| =~\infty$ for every $ 1\leq i \leq n-1$. Suppose further $p_0(x)$ is a real polynomial having distinct roots $\mu_1 < \eta <\mu_2 < \cdots < \mu_{n-1},$ i.e., $p_0(x) = (x-\eta)f_0(x).$ Then we also have that
\[\lim_{n\to\infty} |\mu_{i+1}[(1-D)^m p_0] - \mu_{i}[(1-D)^m p_0]\big|=\infty,~\]
for every $1\leq i \leq n-1.$

\end{lemma}
\proof
We first show that
\[|\mu_{2}[(1-D)^m p_0] - \mu_{1}[(1-D)^m p_0]| \geq |\lambda_{2}[(1-D)^m f_0] - \lambda_{1}[(1-D)^m f_0]|,~\forall m\in\mathbb{N}.\]


Let $\varepsilon > 0$. Now since $\lim_{m\to\infty} |\lambda_{2}[(1-D)^m f_0] - \lambda_{1}[(1-D)^m f_0]| = \infty$, there exist $M\in\mathbb{N}$ such that $|\mu_{2}[(1-D)^m p_0] - \mu_{1}[(1-D)^m p_0]| \geq \frac{1}{\varepsilon},~\forall m\geq M$. Pick $\delta > 0$ The roots of $(1-D)^{m+1} p_0$ satisfy
\[\sum_1^n \frac{1}{x-\mu^{(m)}_i} < \varepsilon \]
\qed

\end{comment}


\begin{lemma}
\label{minimal gap shall approach}
 Consider a sequence $\{f_k\}_1^\infty$ of real-rooted polynomials defined by $f_k(x) = (1-D)^k f_0(x)$, where $f_0(x)$ is a real-rooted polynomial of degree~$m$. Suppose that for some $n\in\mathbb{N}$, there exist an index $1\leq i\leq m-1$ satisfying $d_{i-1}(f_n) \geq L$ and $d_{i+1}(f_n) \geq L$ for some real number $L$ bigger than $m$. Then we have that $\lambda^{(n+1)}_i$ is smaller than the smaller root of
 \[\frac{i-1}{L}+\frac{1}{x-\lambda^{(n)}_i} + \frac{1}{x-\lambda^{(n)}_{i+1}} = 1.\]
 Similarly, $\lambda^{(n+1)}_{i+1}$ is bounded from bellow by the bigger root of
 \[\frac{n-(i+1)}{L}+\frac{1}{x-\lambda^{(n)}_i} + \frac{1}{x-\lambda^{(n)}_{i+1}} = 1.\]
\end{lemma}
\proof
We only need to show the first bound since the second one can be proven analogously. We know that $\lambda^{(n+1)}_i\in(\lambda^{(n)}_i,\lambda^{(n)}_{i+1})$. Clearly, for every $x\in (\lambda^{(n)}_i,\lambda^{(n)}_{i+1})$ we have
\[\sum^{i-1}_{k=1} \frac{1}{x-\lambda^{(n)}_k} + \frac{1}{x-\lambda^{(n)}_i} + \frac{1}{x-\lambda^{(n)}_{i+1}} + \sum^n_{k = i+2} \frac{1}{x-\lambda^{(n)}_k} < \frac{i-1}{L}+\frac{1}{x-\lambda^{(n)}_i} + \frac{1}{x-\lambda^{(n)}_{i+1}}.\]
However, the inequality $L > m,$ implies $\frac{i-1}{L}<1$ and thus all roots of $\frac{i-1}{L}+\frac{1}{x-\lambda^{(n)}_i} + \frac{1}{x-\lambda^{(n)}_{i+1}} = 1$ are to the right of $\lambda^{(n)}_i$, or, more precisely, there is one root in each of the intervals $(\lambda^{(n)}_i,\lambda^{(n)}_{i+1})$ and $(\lambda^{(n)}_{i+1},+\infty)$. Now the claim follow since $\frac{1}{x-\lambda^{(n)}_i} + \frac{1}{x-\lambda^{(n)}_{i+1}}$ is monotonically decreasing in $(\lambda^{(n)}_i,\lambda^{(n)}_{i+1})$.
\qed
Now we have all the tools necessary for proving

\begin{proposition}
\label{minimal gap is unbounded}
Suppose there is a sequence $\{f_k\}_1^\infty$ of real-rooted polynomials defined by $f_k(x) = (1-D)^k f_0(x)$, where $f_0(x)$ is real-rooted of degree $m$. Then
\[\lim_{n\to\infty} \min_{i \in [m]} d_i (f_n) = \infty\]
holds true.
\end{proposition}
\proof Lemma~\ref{minimal gap increases} asserts the existence of such a limit. For the sake of contradiction, assume it is finite, say $l$.

We first analyze the behavior of the gaps between nonconsecutive roots of the $f_k(x)$ as $k$ increases. Applying Lemma~\ref{degree two avg distance} in combination with Lemma~\ref{degree two - slowest movement} easily show that all such gaps increase unboundedly and go to infinity. In particular, we have that
\[|\lambda^{(n)}_{i+2} - \lambda^{(n)}_i|\to \infty\text{~as~$n\to\infty$}\]
for every $1\leq i \leq n-2$.

Pick arbitrarily large number $L\in\mathbb{R}$ and some $\varepsilon > 0$. Then there exist a big enough $M$ so that for all $n\geq M$ we have that all double gaps satisfy $|\lambda^{(n)}_{i+2} - \lambda^{(n)}_i| > L + l$ and also $\min_{1\leq i \leq m} d_i (f_n)> l - \varepsilon$. Denote by $\{j_k\}_{n}^{\infty}$ the sequence of indexes such that $d_{j_n}:=\min_{1\leq i \leq m} d_i (f_n)$ for $n\geq M$. Since both distances $|\lambda^{(n)}_{j_n+1} - \lambda^{(n)}_{j_n-1}|$ and $|\lambda^{(n)}_{j_n+2} - \lambda^{(n)}_{j_n}|$ are bigger than $L + l$, we have that $d_{j_n-1}> L$ and $d_{j_n+1} > L$ for all $n\geq M$.

This means that we are in the conditions of Lemma~\ref{minimal gap shall approach}, which gives us a lower bound of the $d_{j_{n+1}}$ for every $n\geq M$. However, note that if $L$ is big enough, the quantities $\frac{i-1}{L}$ and $\frac{n-(i+1)}{L}$ from Lemma~\ref{minimal gap shall approach} become negligibly small. Therefore, by continuity, for every $\delta > 0$, there is $M$ such that the corresponding $L$ is big enough such that the roots $\lambda^{(n+1)}_{j_n}$ and $\lambda^{(n+1)}_{j_n}$ are both within $\frac{\delta}{2}$ of the roots $\mu^{(n)}_1,~\mu^{(n)}_2$ of $\frac{1}{x-\lambda^{(n)}_{j_n}} + \frac{1}{x-\lambda^{(n)}_{j_n+1}} = 1,~\forall n\geq M$. In particular, $|\lambda^{(n+1)}_{j_n+1} - \lambda^{(n+1)}_{j_n}|>|\mu^{(n)}_1 - \mu^{(n)}_2| - \delta,~\forall n\geq M$.

Recall that $|\mu^{(n)}_1 - \mu^{(n)}_2|= \sqrt{|\lambda^{(n)}_{j_n+1} - \lambda^{(n)}_{j_n}|^2 + 4}$. Now, since $\varepsilon > 0$ was picked arbitrarily, we might assume it is small enough to supply that $\sqrt{|\lambda^{(n)}_{j_n+1} - \lambda^{(n)}_{j_n}|^2 + 4} >l + \frac{\sqrt{l^2 -4} - l}{2}$ for all $n\ge M$ and so $|\mu^{(n)}_1 - \mu^{(n)}_2|\ge \frac{\sqrt{l^2 -4} - l}{2}$ for all $n\ge M$. Now, the choice of $\delta$ was independent of $\varepsilon$, so we pick $0< \delta <\frac{\sqrt{l^2 -4} - l}{2}$. This evidently leads to a contradiction with $l = \sup_{n\ge 0} (\min_{i \in [m]} d_i (f_n))$. Hence, the assumption that the limit $l$ is finite is wrong and thus $\lim_{n\to\infty} \min_{i \in [m]} d_i (f_n) = \infty$, as desired. \qed

\subsection{Some additions/comments}
First, another lemma that I think can be useful elsewhere as well:
\begin{lemma}
	Suppose $f$ is a polynomial whose $i$th smallest root $\mu_i$ is moved (necessarily to the right) a distance $\delta$ by the $(1-D)$ operator. Given $\mu \in \R$, consider the polynomial $\tilde{f} = (x-\mu)f$. Then
	\begin{enumerate}
		\item if $\mu \leq \mu_i$, then $\lambda_{i+1}\big[(1-D)\tilde{f}\big] \geq \mu_i + \delta$ (note that since a root has been added to the left, the $(i+1)$th root of $\tilde{f}$ is the root which intuitively corresponds to the $i$th root of $f$), and
		\item if $\mu > \mu_i$, then $\lambda_{i}\big[(1-D)\tilde{f}\big] \leq \mu_i + \delta$.
	\end{enumerate}
	Loosely speaking, the added root has the effect of `repelling' the old roots of $(1-D)f$.
\end{lemma}
\begin{proof}
	First consider the second case, $\mu > \mu_i$. If $\mu \leq \mu_i + \delta$, then the result follows immediately from interlacing. \\
	If instead $\mu > \mu_i + \delta$, we can use the logarithmic derivative to get the bound we want. If $\delta=0$ then $\mu_i$ is a root of multiplicity and adding the root $\mu$ will not change how much it moves. Assume instead then that $\delta>0$. Then $\mu_i+\delta$ is a root of $\frac{f^\prime}{f}$, i.e.
		\[ \sum_{j=1}^n \frac{1}{(\mu_i+\delta)-\mu_j} = 1. \]
	We would like to find the root of $(1-D)\tilde{f}$, that is the solution of
		\[ \sum_{j=1}^n \frac{1}{x-\mu_j} + \frac{1}{x-\mu} = 1, \]
	that is closest to the right of $\mu_i$. We know that for $x=\mu_i+\delta$ the sum is equal to 1 and the second term is negative, so the left hand side is less than 1. It is also apparent that the sum is much greater than 1 for $x$ just to the right of $\mu_i$, and continuous for $x \in (\mu_i, \mu_i+\delta)$. Therefore by the intermediate value theorem there is a solution in $(\mu_i, \mu_i+\delta)$ and the result immediately follows. \\
	The proof for $\mu<\mu_i$ is essentially the same.
\end{proof}

The following lemma easily follows:

\begin{lemma}
	Appending a root between two (not necessarily adjacent) roots increases the change in their difference upon applying $1-D$.
\end{lemma}
For example, if the gap between the roots of $(x-1)(x-3)$ changes by $\delta$, then the gap between the outer two roots of $(x-1)(x-2)(x-3)$ will change by more than $\delta$.

We already know that the gap between the largest and smallest roots is nondecreasing. The previous lemma can be used to show that it approaches infinity:

\begin{lemma}
\label{lem:maxgapinf}
	For any polynomial $f$,
	 \[ \lim_{n\to\infty} \lambda_{\max}\big[(1-D)^n f\big]-\lambda_{\min}\big[(1-D)^n f\big] = \infty. \]
\end{lemma}
\begin{proof}
	Because the quantity of interest is nondecreasing, it is enough to show that it has no finite limit. Suppose it did approach some number $l < \infty$. Then there is some number of iterations of $1-D$ after which the difference between the maximum and minimum roots is arbitrarily close to $l$. Suppose a polynomial with only two roots separated by $l$ has its gap increased by $\delta$. Then if the two roots are closer than $l$ and more roots are appended in between, the gap will still increase by at least $\delta$. Therefore if the gap between extremal roots is at least $l - \delta/2$, then after applying $1-D$ it will be at least $l + \delta/2$. This is a contradiction, so it must be that the gap is unbouded.
\end{proof}

Now we only need one more lemma, after which the theorem will follow by induction. First a definition: by `$n$-gap' we mean a gap between two roots that have $n-1$ other roots between them. For example $\mu_4-\mu_1$ is a 3-gap, and the difference between the minimum and maximum roots of a degree $m$ polynomial is an $(m-1)$-gap. Now the lemma:

\begin{lemma}
\label{lem:gapinduction}
	Suppose the $n$-gaps of a given polynomial approach infinity. Then the $n-1$ gaps do as well.
\end{lemma}
\begin{proof}
	Suppose the $(n-1)$-gaps of a polynomial $f$ of degree $m$ do \emph{not} approach infinity. Then the $\liminf$ of the smallest $(n-1)$-gap is finite, say $l$. We can write this as
		\[ \liminf_{i\to\infty} G_{n-1}\big[(1-D)^i f\big] = l \]
	where $G_n[f]$ is the size of the smallest $n$-gap of $f$. Choose $\e$ so that two (isolated) roots with separation less than $l+\e$ will have their separation increased by at least $3\e$. Choose $L$ large enough that
	\begin{enumerate}
		\item If an $(n-1)$-gap is at most $l+m+\e$ wide, and all roots outside the gap are at least $L$ away, then the size of the gap increases; and
		\item If an $(n-1)$-gap is at most $l+\e$ wide, and all roots outside the gap are at least $L$ away, then the outside roots' effect on the change in gap size is at most $\e$.
	\end{enumerate}
	By the assumption that the $n$-gaps approach infinity, there exists $N$ so that for all $i>N$
		\[ G_n\big[(1-D)^i f\big] > L + l + m + \e. \]
	By the assumption that the $\liminf$ of the smallest $(n-1)$-gap is $l$, there exist arbitrarily large $j$ such that
		\[ l-\e < G_{n-1}\big[ (1-D)^j f\big] < l+\e. \]
	Choose such a $j$ which is greater than $N$. Due to the choice of $\e$ and $N$, we then have
		\[ G_{n-1}\big[ (1-D)^{j+1} f\big] > l+\e. \]
	For all further iterations, we have that an $(n-1)$-gap either increases, or if it is larger than $l+m+\e$, it may decrease. However, it may only decrease by at most $m$. Therefore the size of an $(n-1)$-gap will never drop below $l+\e$, contradicting that $l$ is the $\liminf$. It must then be that the $\liminf$ is infinite, so the size of the smallest $(n-1)$-gap approaches infinity.
\end{proof}

Lemmas \ref{lem:maxgapinf} and \ref{lem:gapinduction} together immediately give
\begin{theorem}
	If $f$ is any polynomial, the difference between any pair of roots of $(1-D)^n f$ approaches infinity as $n$ approaches infinity.
\end{theorem}

However, in a sense the roots tend to converge on their average, as given by the following theorem.
\begin{theorem}
	If $f$ is any polynomial, then for any $i$ the ratio of the $i$th root of $(1-D)^nf$ to the average root approaches $1$ as $n$ approaches infinity.
\end{theorem}
\begin{proof}
	Because the coefficients of a polynomial are the symmetric polynomials in its roots, given $f(x)=\sum a_i x^i$ with roots $\mu_1,\cdots,\mu_m$ and average root $\mu_{\avg}$ it follows that the polynomial
		\[ \sum_{i=0}^m \frac{a_i}{\mu_{\avg}^{m-i}}x^i \]
	has roots $\mu_i/\mu_{\avg}$. Therefore we want to show that as $n$ approaches infinity, the polynomial obtained by dividing the $i$th coefficient of $(1-D)^n f$ by the $(m-i)$th power of its average root approaches $(x-1)^m$.
	
	$1-D$ always increases the average root by 1, so
		\[ \mu_{\avg}^{(n)} = \mu_{\avg}+n. \]
	We now just need to calculate the coefficients $a_i^{(n)}$. It will ease the calculations if we first notice that the coefficients do not need to be calculated exactly: since the denominator will be degree $m-i$ in $n$ we only need to keep track of terms that are at least that big. We may also assume that $f$ is monic.
	
	Using these simplifying observations we will prove by induction on $i$ that
		\[ a_{i}^{(n)} = (-1)^{m-i}\binom{m}{i}n^{m-i} + \text{lower order terms,} \]
	so that the result will immediately follow.
	
	For $i=m$, the equation is trivial because we have assumed $f$ is monic:
		\[ a_m^{(n)} = 1. \]
	This provides the base case for our induction.
	
	Now suppose the equation holds for some $i$; we will show it also holds for $i-1$. We have
		\[ a_{i-1}^{(n)} = a_{i-1}^{(n-1)} - i a_i^{(n-1)} \]
	which by the inductive assumption is equal to
		\[ a_{i-1}^{(n-1)} - i (-1)^{m-i}\binom{m}{i}n^{m-i} + \text{lower order terms}. \]
	Repeating this $n-1$ times gives that
		\[ a_{i-1}^{(n)} = a_{i-1} - \sum_{j=1}^n i (-1)^{m-i} \binom{m}{i} j^{m-i} + \text{lower order terms} \]
		\[ = i (-1)^{m-i+1} \binom{m}{i} \sum_{j=1}^n j^{m-i} + \text{lower order terms.} \]
	The exact formula for this sum is
		\[ \sum_{j=1}^n j^{m-i} = \frac{1}{m-i+1}\sum_{k=0}^{m-i} \binom{m-i+1}{k}B_k n^{m-i+1-k} \]
	but again we are only interested in the largest degree term, where $k=0$:
		\[ \frac{1}{m-i+1}n^{m-i+1}. \]
	Therefore we now have
		\[ a_{i-1}^{(n)} = (-1)^{m-(i-1)} i \frac{m!}{i!(m-i)!} \cdot \frac{1}{m-i+1} n^{m-i+1} + \text{lower order terms} \]
		\[ = (-1)^{m-(i-1)} \binom{m}{i-1} n^{m-(i-1)} + \text{lower order terms}\]
	which is what we wanted.
\end{proof}

\begin{thebibliography}{1}

  \bibitem{Borcea1} J. Borcea, and P. Branden {\em The Lee-Yang and Polya-Schur programs. I. Linear operators preserving stability.} Invent. Math. 177, pp. 541-569, 2009.

 \bibitem{Borcea2} J. Borcea, and P. Branden {\em The Lee-Yang and Polya-Schur programs. II. Theory of stable polynomials and applications} Comm. Pure Appl. Math. 62, pp. 1595-1631, 2009.

   \bibitem{Spielman} A. Marcus, D. Spielman, and N. Srivastava. {\em Ramanujan graphs and the solution of the Kadison-Singer problem.} Proceedings of ICM, Seoul, 2014.
\bibitem{diffspacing}
 Farmer, D.W. and Rhoades, R.C.
 \emph{Differentiation Evens out Zero Spacings}.
 arXiv:math/0310252.
   \bibitem{Fisk} S. Fisk. {\em Polynomials, roots, and interlacing.} available at \url{http://arxiv.org/abs/math/0612833}, 2008.
  \end{thebibliography}

\end{document}

