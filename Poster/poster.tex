\documentclass[landscape,final,a0paper,fontscale=0.29]{baposter}

\usepackage[english]{babel}
\usepackage{amsthm}

\usepackage{calc}
\usepackage{graphicx}
\usepackage{amsmath}
\usepackage{amssymb}
\usepackage{relsize}
\usepackage{multirow}
%\usepackage{rotating}
\usepackage{bm}
\usepackage{url}
\usepackage{tikz}
\usetikzlibrary{calc}

%Useful Theorem Environments
\newtheorem{theorem}{Theorem}
\newtheorem{corollary}[theorem]{Corollary}
\newtheorem{lemma}[theorem]{Lemma}
\newtheorem{proposition}[theorem]{Proposition}
\newtheorem{definition}[theorem]{Definition}
\newtheorem{warning}[theorem]{Warning}
\newtheorem{questions}{Question}
\newtheorem{conjecture}[theorem]{Conjecture}
\newtheorem{assumption}{Assumption}
\newtheorem{example}[theorem]{Example}
\newtheorem{construction}[theorem]{Construction}
\newtheorem{quasi-theorem}[theorem]{Quasi-Theorem}

\newtheorem{rem1}[theorem]{Remark}
\newenvironment{remark}{\begin{rem1}\em}{\end{rem1}}

\newtheorem{not1}[theorem]{Notation}
\newenvironment{notation}{\begin{not1}\em}{\end{not1}}

\usepackage{mathptmx} %make math font slightly better

\usepackage{graphicx}
\usepackage{multicol}
\usepackage{palatino}

\newcommand{\captionfont}{\footnotesize}

\graphicspath{{images/}{../images/}}
\usetikzlibrary{calc}

\newcommand{\SET}[1]  {\ensuremath{\mathcal{#1}}}
\newcommand{\MAT}[1]  {\ensuremath{\boldsymbol{#1}}}
\newcommand{\VEC}[1]  {\ensuremath{\boldsymbol{#1}}}
\newcommand{\Video}{\SET{V}}
\newcommand{\video}{\VEC{f}}
\newcommand{\track}{x}
\newcommand{\Track}{\SET T}
\newcommand{\LMs}{\SET L}
\newcommand{\lm}{l}
\newcommand{\PosE}{\SET P}
\newcommand{\posE}{\VEC p}
\newcommand{\negE}{\VEC n}
\newcommand{\NegE}{\SET N}
\newcommand{\Occluded}{\SET O}
\newcommand{\occluded}{o}

\renewcommand{\rmdefault}{ptm} %change all font to Times
\usepackage[english]{babel}

%\rmfamily


\setlength{\columnsep}{1.5em}
\setlength{\columnseprule}{0mm}

%%%%%%%%%%%%%%%%%%%%%%%%%%%%%%%%%%%%%%%%%%%%%%%%%%%%%%%%%%%%%%%%%%%%%%%%%%%%%%%%
% Save space in lists. Use this after the opening of the list
%%%%%%%%%%%%%%%%%%%%%%%%%%%%%%%%%%%%%%%%%%%%%%%%%%%%%%%%%%%%%%%%%%%%%%%%%%%%%%%%
\newcommand{\compresslist}{%
\setlength{\itemsep}{1pt}%
\setlength{\parskip}{0pt}%
\setlength{\parsep}{0pt}%
}

%%%%%%%%%%%%%%%%%%%%%%%%%%%%%%%%%%%%%%%%%%%%%%%%%%%%%%%%%%%%%%%%%%%%%%%%%%%%%%
%%% Begin of Document
%%%%%%%%%%%%%%%%%%%%%%%%%%%%%%%%%%%%%%%%%%%%%%%%%%%%%%%%%%%%%%%%%%%%%%%%%%%%%%

\begin{document}

%%%%%%%%%%%%%%%%%%%%%%%%%%%%%%%%%%%%%%%%%%%%%%%%%%%%%%%%%%%%%%%%%%%%%%%%%%%%%%
%%% Here starts the poster
%%%---------------------------------------------------------------------------
%%% Format it to your taste with the options
%%%%%%%%%%%%%%%%%%%%%%%%%%%%%%%%%%%%%%%%%%%%%%%%%%%%%%%%%%%%%%%%%%%%%%%%%%%%%%
% Define some colors

%\definecolor{lightblue}{cmyk}{0.83,0.24,0,0.12}
%\definecolor{lightblue}{rgb}{0.145,0.6666,1}
\definecolor{lightblue}{rgb}{0.115,0.7266,0.7}

% Draw a video
\newlength{\FSZ}
\newcommand{\drawvideo}[3]{% [0 0.25 0.5 0.75 1 1.25 1.5]
   \noindent\pgfmathsetlength{\FSZ}{\linewidth/#2}
   \begin{tikzpicture}[outer sep=0pt,inner sep=0pt,x=\FSZ,y=\FSZ]
   \draw[color=lightblue!50!black] (0,0) node[outer sep=0pt,inner sep=0pt,text width=\linewidth,minimum height=0] (video) {\noindent#3};
   \path [fill=lightblue!50!black,line width=0pt]
     (video.north west) rectangle ([yshift=\FSZ] video.north east)
    \foreach \x in {1,2,...,#2} {
      {[rounded corners=0.6] ($(video.north west)+(-0.7,0.8)+(\x,0)$) rectangle +(0.4,-0.6)}
    }
;
   \path [fill=lightblue!50!black,line width=0pt]
     ([yshift=-1\FSZ] video.south west) rectangle (video.south east)
    \foreach \x in {1,2,...,#2} {
      {[rounded corners=0.6] ($(video.south west)+(-0.7,-0.2)+(\x,0)$) rectangle +(0.4,-0.6)}
    }
;
   \foreach \x in {1,...,#1} {
     \draw[color=lightblue!50!black] ([xshift=\x\linewidth/#1] video.north west) -- ([xshift=\x\linewidth/#1] video.south west);
   }
   \foreach \x in {0,#1} {
     \draw[color=lightblue!50!black] ([xshift=\x\linewidth/#1,yshift=1\FSZ] video.north west) -- ([xshift=\x\linewidth/#1,yshift=-1\FSZ] video.south west);
   }
   \end{tikzpicture}
}

\hyphenation{resolution occlusions}
%%
\begin{poster}%
  % Poster Options
  {
  % Show grid to help with alignment
  grid=false,
  % Column spacing
  colspacing=1em,
  columns=4,
  % Color style
  bgColorOne=white,
  bgColorTwo=white,
  borderColor=black,
  headerColorOne=black,
  headerColorTwo=lightblue,
  headerFontColor=white,
  boxColorOne=white,
  boxColorTwo=lightblue,
  % Format of textbox
  textborder=roundedleft,
  % Format of text header
  eyecatcher=true,
  headerborder=closed,
  headerheight=0.1\textheight,
%  textfont=\sc, An example of changing the text font
  headershape=roundedright,
  headershade=shadelr,
  headerfont=\Large\bf\textsc, %Sans Serif
  textfont={\setlength{\parindent}{1.5em}},
  boxshade=plain,
%  background=shade-tb,
  background=plain,
  linewidth=2pt
  }
  % Eye Catcher
    %{\includegraphics[height=8em][bb=0 0 360 360]{images/yale}}
  % Title
  {\bf\textsc{Poster Title}\vspace{0.5em}}
  % Authors
  {\textsc{ Student Names\\ Advised by:}}
  % University logo
  {% The makebox allows the title to flow into the logo, this is a hack because of the L shaped logo.

\setlength{\fboxsep}{0pt}%
\setlength{\fboxrule}{3pt}%
  }

%%%%%%%%%%%%%%%%%%%%%%%%%%%%%%%%%%%%%%%%%%%%%%%%%%%%%%%%%%%%%%%%%%%%%%%%%%%%%%
%%% Now define the boxes that make up the poster
%%%---------------------------------------------------------------------------
%%% Each box has a name and can be placed absolutely or relatively.
%%% The only inconvenience is that you can only specify a relative position
%%% towards an already declared box. So if you have a box attached to the
%%% bottom, one to the top and a third one which should be in between, you
%%% have to specify the top and bottom boxes before you specify the middle
%%% box.
%%%%%%%%%%%%%%%%%%%%%%%%%%%%%%%%%%%%%%%%%%%%%%%%%%%%%%%%%%%%%%%%%%%%%%%%%%%%%%
    %
    % A coloured circle useful as a bullet with an adjustably strong filling
    \newcommand{\colouredcircle}{%
      \tikz{\useasboundingbox (-0.2em,-0.32em) rectangle(0.2em,0.32em); \draw[draw=black,fill=lightblue,line width=0.03em] (0,0) circle(0.18em);}}


%%%%%%%%%%%%%%%%%%%%%%%%%%%%%%%%%%%%%%%%%%%%%%%%%%%%%%%%%%%%%%%%%%%%%%%%%%%%%%
%%%%%%%%%%%%%%%%%%%%%%%%%%%%%%%%%%%%%%%%%%%%%%%%%%%%%%%%%%%%%%%%%%%%%%%%%%%%%%
  %%%%%%%%%%%%%%%%%%%%%%%%%%%%%%%%%%%%%%%%%%%%%%%%%%%%%%%%%%%%%%%%%%%%%%%%%%%%%%

%BACKGROUND SECTION
\headerbox{Background}{name=BG,column=0,row=0,span=4}{
\begin{multicols}{2}
\large
Real-rooted polynomials are these polynomials who have only real roots. We say that a linear transformation $T:\mathbb{R}[x]\to\mathbb{R}[x]$ \emph{preserves real-rootedness}, if $T(f)$ is real-rooted, whenever $f\in\mathbb{R}[x]$ is real-rooted.  Such transformations have been object of many recent research and their theory has had a significant progress in recent years. Notably, Borcea and Br\"{a}nden found their complete characterization. Nevertheless, their result does not shed any light on where on the real line do the roots of $T(f)$ lie as some dependance on the roots of $f$. We analyze this question focusing on two real-rootedness preserves. Most of the results, however, can be extended to a big classes of real-rootedness preservers.
\end{multicols}
}

%%%%%%%%%%%%%%%%%%%%%%%%%%%%%%%%%%%%%%%%%%%%%%%%%%%%%%%%%%%%%%%%%%%%%%%%%%%%%%%

%FIRST MULTICOLUMN
\headerbox{Introduction to making a poster}{name=TL,column=0,span=2,below=BG}{
%%%%%%%%%%%%%%%%%%%%%%%%%%%%%%%%%%%%%%%%%%%%%%%%%%%%%%%%%%%%%%%%%%%%%%%%%%%%%%
\begin{multicols}{2}
\large
\noindent \newtheorem *{question}{Question}
\begin  {question}
The motivating question for my project is: ``can I make a good poster?"
\end {question}

\noindent A poster should not contain too much text -- this is not a good place for your proofs, and pseudocode, it is a gadget that has just enough to point at so you can effectively and crisply explain the essence of your ideas: the problem, some examples, perhaps counterexamples, an approach to a solution, good illustrations.

\begin{definition}
A poster $P$ is said to be \textit{good} if it satisfies all the properties of being good, and is a poster.
\end{definition}

Our first main result is an existence result.
\begin{theorem}
There exist at most finitely many good posters.
\end{theorem}


\end{multicols}
}

%%%%%%%%%%%%%%%%%%%%%%%%%%%%%%%%%%%%%%%%%%%%%%%%%%%%%%%%%%%%%%%%%%%%%%%%%%%%%%
%%%%%%%%%%%%%%%%%%%%%%%%%%%%%%%%%%%%%%%%%%%%%%%%%%%%%%%%%%%%%%%%%%%%%%%%%%%%%%
\headerbox{My Generation}{name=BL,column=0,row=3,span=2,below=TL}{
\begin{multicols}{2}
\large
People try to put us d-down (Talkin' 'bout my generation)\\
Just because we get around (Talkin' 'bout my generation)\\
Things they do look awful c-c-cold (Talkin' 'bout my generation)\\
I hope I die before I get old (Talkin' 'bout my generation)\\
\bigskip

This is my generation\\
This is my generation, baby\\
\bigskip

Why don't you all f-fade away (Talkin' 'bout my generation)\\
And don't try to dig what we all s-s-say (Talkin' 'bout my generation)\\
I'm not trying to cause a big s-s-sensation (Talkin' 'bout my generation)\\
I'm just talkin' 'bout my g-g-g-generation (Talkin' 'bout my generation)\\
\bigskip

This is my generation\\
This is my generation, baby\\

\end{multicols}
}

%END OF THIS MULTICOLUMN

%%%%%%%%%%%%%%%%%%%%%%%%%%%%%%%%%%%%%%%%%%%%%%%%%%%%%%%%%%%%%%%%%%%%%%%%%%%%%%
%%%%%%%%%%%%%%%%%%%%%%%%%%%%%%%%%%%%%%%%%%%%%%%%%%%%%%%%%%%%%%%%%%%%%%%%%%%%%%
    \headerbox{The \small{$T_!:x \to \frac{1}{k!}x^k$} operator.}{name=TR,row=1,column=2,span=2, below=BG}{
%%%%%%%%%%%%%%%%%%%%%%%%%%%%%%%%%%%%%%%%%%%%%%%%%%%%%%%%%%%%%%%%%%%%%%%%%%%%%%
%\begin{multicols}{1}
\large
\noindent
Example:
\[x^4 + 7x^3 + 3x^2 + 2x + 1 \xrightarrow{T_!} \frac{1}{24}x^4 + \frac{7}{6}x^3 + \frac{3}{2}x^2 + 2x + 1\]
\begin{itemize}
 \item Preserves real-rootedness. In fact, if $p(x)$ has real roots $\lambda_1,\cdots,\lambda_n$, then
$T!(p) = [\prod_{\lambda_i \neq 0} (\frac{1}{\lambda_i} - D) \prod_{\lambda_i = 0} D]p(x).$
 \item Links ordinary and exponential functions (used for counting objects in combinatorics):
$\sum_{n\ge 0} a_n x^n \xrightarrow{T_!} \sum_{n\ge 0} \frac{a_n}{n!} x^n$

\end{itemize}
%\end{multicols}
}


%%%%%%%%%%%%%%%%%%%%%%%%%%%%%%%%%%%%%%%%%%%%%%%%%%%%%%%%%%%%%%%%%%%%%%%%%%%%%%
%%%%%%%%%%%%%%%%%%%%%%%%%%%%%%%%%%%%%%%%%%%%%%%%%%%%%%%%%%%%%%%%%%%%%%%%%%%%%%
\headerbox{\vspace{-1cm}Majorization}{name=MJRZ,column=2,row=3,span=2,below=TR}{
\begin{multicols}{2}
\noindent
\large
If the polynomial $p(x)$ has real roots $\lambda_1\leq\cdots\leq\lambda_n$, we associate it with the weakly-increasing vector $\lambda = (\lambda_1,\cdots,\lambda_n)$ of its roots. Suppose $x = (x_1,\cdots, x_n)$ and $y = (y_1,\cdots, y_n)$ are weakly increasing vectors in $\mathbb{R}^n$, then $x$ \emph{majorizes} $y$ (denoted by $x \succ y$) if $\sum_{i=1}^n x_i = \sum_{i=1}^n y_i$ and $\sum_{i=k}^n x_i \geq \sum_{i=}^n y_i$ for $2 \leq k\leq n.$ Given two polynomials $p(x)$ and $q(x)$, we say $p(x)$ \emph{majorizes} $q(x)$, $p \succ q$, if $\lambda \succ \mu$, where $\lambda$ and $\mu$ are the weakly-increasing vectors, associated to $p$ and $q$, respectively.
\begin{theorem}{(Br\"{a}nden)}
The linear operator $T_!:~x^k\to\frac{1}{k!}x^k$ preserves majorization, i.e., $T_!(p) \succ T_!(q),$ whenever $p \succ q$.
\end{theorem}
 \begin{corollary}
  $p(x) \succ q(x) \Rightarrow \lambda_{max} T_!(p) \geq \lambda_{max} T_!(q)$
 \end{corollary}
\end{multicols}
}

\headerbox{Movement of maximum root}{name=MRT,column=2,row=4,span=2,below=MJRZ}{
By Vieta's formulas the roots of $p(x) = \sum_{i=1} a_i x^i$ have sum $\sum_1^n \lambda_i = -\frac{a_n}{a_{n-1}}$.
\begin{theorem} For $p(x) = \sum_{i=1} a_i x^i$, we have
\[\lambda_{max}(p(x)) \geq \frac{ \lambda_{max} (T_!(p))}{\lambda_{max}(L_n(x))} \geq -\frac{a_{n-1}}{n a_n},\]
 where $L_n(x)$ is the $n$-th Laguerre polynomial
\end{theorem}
}
%%%%%%%%%%%%%%%%%%%%%%%%%%%%%%%%%%%%%%%%%%%%%%%%%%%%%%%%%%%%%%%%%%%%%%%%%%%%%%
%%%%%%%%%%%%%%%%%%%%%%%%%%%%%%%%%%%%%%%%%%%%%%%%%%%%%%%%%%%%%%%%%%%%%%%%%%%%%%
\headerbox{References}{name=Ref,column=0,row=5,span=4,below=MRT}{
\noindent

 S. Fisk. {\em Polynomials, roots, and interlacing.} 2008.

 A. Marcus, D. Spielman, and N. Srivastava. {\em Ramanujan graphs and the solution of the Kadison-Singer problem.} Proceedings of ICM, Seoul, 2014.

 J. Borcea, and P. Br\"{a}nden {\em Hyperbolicity preservers and majorization.} 2010.

}


\end{poster}

\end{document}
