\documentclass[landscape,final,a0paper,fontscale=0.27]{baposter}

\usepackage[english]{babel}
\usepackage{amsthm}

\usepackage{calc}
\usepackage{graphicx}
\usepackage{amsmath}
\usepackage{amssymb}
\usepackage{relsize}
\usepackage{multirow}
%\usepackage{rotating}
\usepackage{bm}
\usepackage{url}
\usepackage{tikz}
\usetikzlibrary{calc}

%Useful Theorem Environments
\newtheorem{theorem}{Theorem}
\newtheorem{corollary}[theorem]{Corollary}
\newtheorem{lemma}[theorem]{Lemma}
\newtheorem{proposition}[theorem]{Proposition}
\newtheorem{definition}[theorem]{Definition}
\newtheorem{warning}[theorem]{Warning}
\newtheorem{questions}{Question}
\newtheorem{conjecture}[theorem]{Conjecture}
\newtheorem{assumption}{Assumption}
\newtheorem{example}[theorem]{Example}
\newtheorem{construction}[theorem]{Construction}
\newtheorem{quasi-theorem}[theorem]{Quasi-Theorem}

\newtheorem{rem1}[theorem]{Remark}
\newenvironment{remark}{\begin{rem1}\em}{\end{rem1}}

\newtheorem{not1}[theorem]{Notation}
\newenvironment{notation}{\begin{not1}\em}{\end{not1}}

\usepackage{mathptmx} %make math font slightly better

\usepackage{graphicx}
\usepackage{multicol}
\usepackage{palatino}

\newcommand{\captionfont}{\footnotesize}

\graphicspath{{images/}{../images/}}
\usetikzlibrary{calc}

\newcommand{\SET}[1]  {\ensuremath{\mathcal{#1}}}
\newcommand{\MAT}[1]  {\ensuremath{\boldsymbol{#1}}}
\newcommand{\VEC}[1]  {\ensuremath{\boldsymbol{#1}}}
\newcommand{\Video}{\SET{V}}
\newcommand{\video}{\VEC{f}}
\newcommand{\track}{x}
\newcommand{\Track}{\SET T}
\newcommand{\LMs}{\SET L}
\newcommand{\lm}{l}
\newcommand{\PosE}{\SET P}
\newcommand{\posE}{\VEC p}
\newcommand{\negE}{\VEC n}
\newcommand{\NegE}{\SET N}
\newcommand{\Occluded}{\SET O}
\newcommand{\occluded}{o}

\renewcommand{\rmdefault}{ptm} %change all font to Times
\usepackage[english]{babel}

%\rmfamily


\setlength{\columnsep}{1.5em}
\setlength{\columnseprule}{0mm}

%%%%%%%%%%%%%%%%%%%%%%%%%%%%%%%%%%%%%%%%%%%%%%%%%%%%%%%%%%%%%%%%%%%%%%%%%%%%%%%%
% Save space in lists. Use this after the opening of the list
%%%%%%%%%%%%%%%%%%%%%%%%%%%%%%%%%%%%%%%%%%%%%%%%%%%%%%%%%%%%%%%%%%%%%%%%%%%%%%%%
\newcommand{\compresslist}{%
\setlength{\itemsep}{1pt}%
\setlength{\parskip}{0pt}%
\setlength{\parsep}{0pt}%
}

%%%%%%%%%%%%%%%%%%%%%%%%%%%%%%%%%%%%%%%%%%%%%%%%%%%%%%%%%%%%%%%%%%%%%%%%%%%%%%
%%% Begin of Document
%%%%%%%%%%%%%%%%%%%%%%%%%%%%%%%%%%%%%%%%%%%%%%%%%%%%%%%%%%%%%%%%%%%%%%%%%%%%%%

\begin{document}

%%%%%%%%%%%%%%%%%%%%%%%%%%%%%%%%%%%%%%%%%%%%%%%%%%%%%%%%%%%%%%%%%%%%%%%%%%%%%%
%%% Here starts the poster
%%%---------------------------------------------------------------------------
%%% Format it to your taste with the options
%%%%%%%%%%%%%%%%%%%%%%%%%%%%%%%%%%%%%%%%%%%%%%%%%%%%%%%%%%%%%%%%%%%%%%%%%%%%%%
% Define some colors

%\definecolor{lightblue}{cmyk}{0.83,0.24,0,0.12}
%\definecolor{lightblue}{rgb}{0.145,0.6666,1}
\definecolor{lightblue}{rgb}{0.115,0.7266,0.7}

% Draw a video
\newlength{\FSZ}
\newcommand{\drawvideo}[3]{% [0 0.25 0.5 0.75 1 1.25 1.5]
   \noindent\pgfmathsetlength{\FSZ}{\linewidth/#2}
   \begin{tikzpicture}[outer sep=0pt,inner sep=0pt,x=\FSZ,y=\FSZ]
<<<<<<< HEAD
   \draw[color=lightblue!50!black] (0,0) node[outer sep=0pt,inner sep=0pt,text width=\linewidth,minimum height=0] (video) {\noindent#3};
   \path [fill=lightblue!50!black,line width=0pt]
     (video.north west) rectangle ([yshift=\FSZ] video.north east)
=======
   \draw[color=lightblue] (0,0) node[outer sep=0pt,inner sep=0pt,text width=\linewidth,minimum height=0] (video) {\noindent#3};
   \path [fill=lightblue,line width=0pt] 
     (video.north west) rectangle ([yshift=\FSZ] video.north east) 
>>>>>>> origin/master
    \foreach \x in {1,2,...,#2} {
      {[rounded corners=0.6] ($(video.north west)+(-0.7,0.8)+(\x,0)$) rectangle +(0.4,-0.6)}
    }
;
<<<<<<< HEAD
   \path [fill=lightblue!50!black,line width=0pt]
     ([yshift=-1\FSZ] video.south west) rectangle (video.south east)
=======
   \path [fill=lightblue,line width=0pt] 
     ([yshift=-1\FSZ] video.south west) rectangle (video.south east) 
>>>>>>> origin/master
    \foreach \x in {1,2,...,#2} {
      {[rounded corners=0.6] ($(video.south west)+(-0.7,-0.2)+(\x,0)$) rectangle +(0.4,-0.6)}
    }
;
   \foreach \x in {1,...,#1} {
     \draw[color=lightblue] ([xshift=\x\linewidth/#1] video.north west) -- ([xshift=\x\linewidth/#1] video.south west);
   }
   \foreach \x in {0,#1} {
     \draw[color=lightblue] ([xshift=\x\linewidth/#1,yshift=1\FSZ] video.north west) -- ([xshift=\x\linewidth/#1,yshift=-1\FSZ] video.south west);
   }
   \end{tikzpicture}
}

\hyphenation{resolution occlusions}
%%
\begin{poster}%
  % Poster Options
  {
  % Show grid to help with alignment
  grid=false,
  % Column spacing
  colspacing=1em,
  columns=4,
  % Color style
  bgColorOne=white,
  bgColorTwo=white,
  borderColor=black,
  headerColorOne=lightblue,
  headerColorTwo=blue,
  headerFontColor=white,
  boxColorOne=white,
  boxColorTwo=lightblue,
  % Format of textbox
  textborder=rectangle,
  % Format of text header
  eyecatcher=true,
  headerborder=open,
  headerheight=0.1\textheight,
%  textfont=\sc, An example of changing the text font
  headershape=roundedright,
  headershade=plain,
  headerfont=\Large\bf\textsc, %Sans Serif
  textfont={\setlength{\parindent}{1.5em}},
  boxshade=plain,
%  background=shade-tb,
  background=plain,
  linewidth=2pt
  }
  % Eye Catcher
    %{\includegraphics[height=8em][bb=0 0 360 360]{images/yale}}
  % Title
  {\bf\textsc{Properties of Some Real Stability Preservers}\vspace{0.5em}}
  % Authors
  {\textsc{ Stanislav Atanasov and Christopher Shriver\\ Advised by: Anup Rao}}
  % University logo
  {% The makebox allows the title to flow into the logo, this is a hack because of the L shaped logo.

\setlength{\fboxsep}{0pt}%
\setlength{\fboxrule}{3pt}%
  }

%%%%%%%%%%%%%%%%%%%%%%%%%%%%%%%%%%%%%%%%%%%%%%%%%%%%%%%%%%%%%%%%%%%%%%%%%%%%%%
%%% Now define the boxes that make up the poster
%%%---------------------------------------------------------------------------
%%% Each box has a name and can be placed absolutely or relatively.
%%% The only inconvenience is that you can only specify a relative position
%%% towards an already declared box. So if you have a box attached to the
%%% bottom, one to the top and a third one which should be in between, you
%%% have to specify the top and bottom boxes before you specify the middle
%%% box.
%%%%%%%%%%%%%%%%%%%%%%%%%%%%%%%%%%%%%%%%%%%%%%%%%%%%%%%%%%%%%%%%%%%%%%%%%%%%%%
    %
    % A coloured circle useful as a bullet with an adjustably strong filling
    \newcommand{\colouredcircle}{%
      \tikz{\useasboundingbox (-0.2em,-0.32em) rectangle(0.2em,0.32em); \draw[draw=black,fill=lightblue,line width=0.03em] (0,0) circle(0.18em);}}


%%%%%%%%%%%%%%%%%%%%%%%%%%%%%%%%%%%%%%%%%%%%%%%%%%%%%%%%%%%%%%%%%%%%%%%%%%%%%%
%%%%%%%%%%%%%%%%%%%%%%%%%%%%%%%%%%%%%%%%%%%%%%%%%%%%%%%%%%%%%%%%%%%%%%%%%%%%%%
  %%%%%%%%%%%%%%%%%%%%%%%%%%%%%%%%%%%%%%%%%%%%%%%%%%%%%%%%%%%%%%%%%%%%%%%%%%%%%%

%BACKGROUND SECTION
\headerbox{Background}{name=BG,column=0,row=0,span=4}{
\begin{multicols}{2}
\large
Some background information.
\end{multicols}
}

%%%%%%%%%%%%%%%%%%%%%%%%%%%%%%%%%%%%%%%%%%%%%%%%%%%%%%%%%%%%%%%%%%%%%%%%%%%%%%%

<<<<<<< HEAD
%FIRST MULTICOLUMN
\headerbox{Introduction to making a poster}{name=TL,column=0,span=2,below=BG}{
%%%%%%%%%%%%%%%%%%%%%%%%%%%%%%%%%%%%%%%%%%%%%%%%%%%%%%%%%%%%%%%%%%%%%%%%%%%%%%
\begin{multicols}{2}
\large
\noindent \newtheorem *{question}{Question}
\begin  {question}
The motivating question for my project is: ``can I make a good poster?"
\end {question}

\noindent A poster should not contain too much text -- this is not a good place for your proofs, and pseudocode, it is a gadget that has just enough to point at so you can effectively and crisply explain the essence of your ideas: the problem, some examples, perhaps counterexamples, an approach to a solution, good illustrations.

\begin{definition}
A poster $P$ is said to be \textit{good} if it satisfies all the properties of being good, and is a poster.
\end{definition}

Our first main result is an existence result.
\begin{theorem}
There exist at most finitely many good posters.
\end{theorem}
=======
%FIRST MULTICOLUMN  
\headerbox{The $1-D$ Operator}{name=TL,column=0,span=2,below=BG}{
%%%%%%%%%%%%%%%%%%%%%%%%%%%%%%%%%%%%%%%%%%%%%%%%%%%%%%%%%%%%%%%%%%%%%%%%%%%%%%
\begin{multicols}{2}
\large
\noindent The $1-D$ operator, where $D=\frac{d}{dx}$, is a linear map on the set of real-rooted polynomials. It was of particular interest in a recent paper solving the Kadison-Singer conjecture, in part because of its relation to the Laguerre polynomials. We focus primarily on its effect on the differences between the roots of a polynomial.

\end{multicols}
}
>>>>>>> origin/master

\headerbox{Intuitive Picture}{name=TC,column=0,span=2,below=TL}{
%%%%%%%%%%%%%%%%%%%%%%%%%%%%%%%%%%%%%%%%%%%%%%%%%%%%%%%%%%%%%%%%%%%%%%%%%%%%%%
\begin{multicols}{2}
\large
\noindent Given a polynomial $f$, the roots of ${(1-D)f}$ can be understood as the points of equilibrium in a system of forces resulting from placing a unit charge at each of the roots of $f$ and introducing a constant gravitational force in the $-x$ direction. This is shown in the figure below.
\begin{center}
\includegraphics[width=0.9\columnwidth]{images/charges}
\end{center}
The blue line represents the force at $x$ due to only the blue roots, the red line shows the force due to the red root, and the purple line shows their sum. The red arrow shows where the red root repels the old solutions.

This model makes it easy to see the effects of changing the roots of a polynomial, which can be very useful for bounding various quantities. The intuition provided can often be formalized into a more rigorous argument, as was done for Theorem~\ref{thm:rootgaps} below.
\end{multicols}
}

%%%%%%%%%%%%%%%%%%%%%%%%%%%%%%%%%%%%%%%%%%%%%%%%%%%%%%%%%%%%%%%%%%%%%%%%%%%%%%
%%%%%%%%%%%%%%%%%%%%%%%%%%%%%%%%%%%%%%%%%%%%%%%%%%%%%%%%%%%%%%%%%%%%%%%%%%%%%%
\headerbox{Main Results}{name=BL,column=0,row=3,span=2,below=TC}{
\begin{multicols}{2}
\large
\noindent Our main result for this operator is
\begin{theorem}
\label{thm:rootgaps}
	For any real-rooted polynomial $f$, the difference between any two roots of ${(1-D)^n f}$ approaches infinity as $n$ approaches infinity.
\end{theorem}
Interestingly, however, in a more relative sense the roots tend to be close together:

\begin{theorem}
\label{thm:rootratio}
	For any polynomial $f$, the ratio of any two roots of ${(1-D)^n f}$ approaches $1$ as $n$ approaches infinity.
\end{theorem}


\end{multicols}
}

%END OF THIS MULTICOLUMN

%%%%%%%%%%%%%%%%%%%%%%%%%%%%%%%%%%%%%%%%%%%%%%%%%%%%%%%%%%%%%%%%%%%%%%%%%%%%%%
%%%%%%%%%%%%%%%%%%%%%%%%%%%%%%%%%%%%%%%%%%%%%%%%%%%%%%%%%%%%%%%%%%%%%%%%%%%%%%
    \headerbox{The \small{$T_!:x \to \frac{1}{k!}x^k$} operator.}{name=TR,row=1,column=2,span=2, below=BG}{
%%%%%%%%%%%%%%%%%%%%%%%%%%%%%%%%%%%%%%%%%%%%%%%%%%%%%%%%%%%%%%%%%%%%%%%%%%%%%%
%\begin{multicols}{1}
\large
\noindent
Example:
\[x^4 + 7x^3 + 3x^2 + 2x + 1 \xrightarrow{T_!} \frac{1}{24}x^4 + \frac{7}{6}x^3 + \frac{3}{2}x^2 + 2x + 1\]
\begin{itemize}
 \item Preserves real-rootedness. In fact, if $p(x)$ has real roots $\lambda_1,\cdots,\lambda_n$, then
$T!(p) = [\prod_{\lambda_i \neq 0} (\frac{1}{\lambda_i} - D) \prod_{\lambda_i = 0} D]p(x).$
 \item Links ordinary and exponential functions (used for counting objects in combinatorics):
$\sum_{n\ge 0} a_n x^n \xrightarrow{T_!} \sum_{n\ge 0} \frac{a_n}{n!} x^n$
  
\end{itemize}
%\end{multicols}
}


%%%%%%%%%%%%%%%%%%%%%%%%%%%%%%%%%%%%%%%%%%%%%%%%%%%%%%%%%%%%%%%%%%%%%%%%%%%%%%
%%%%%%%%%%%%%%%%%%%%%%%%%%%%%%%%%%%%%%%%%%%%%%%%%%%%%%%%%%%%%%%%%%%%%%%%%%%%%%
\headerbox{Majorization}{name=MJRZ,column=2,row=3,span=2,below=TR}{
\begin{multicols}{2}
\noindent
\large
If $p(x)$ is a real-rooted polynomial with roots $\lambda_1\leq\cdots\leq\lambda_n$ we can associate it with the weakly-increasing vector $\lambda = (\lambda_1,\cdots,\lambda_n)$ of its roots. If $x = (x_1,\cdots, x_n)$ and $y = (y_1,\cdots, y_n)$ are weakly increasing vectors in $\mathbb{R}^n$, then $x$ \emph{majorizes} $y$ (denoted by $x \succ y$) if $\sum_{i=1}^n x_i = \sum_{i=1}^n y_i$ and $\sum_{i=k}^n x_i \geq \sum_{i=}^n y_i$ for $2 \leq k\leq n.$ Given two polynomials $p(x)$ and $q(x)$, we say $p(x)$ \emph{majorizes} $q(x)$, $p \succ q$, if $\lambda \succ \mu$, where $\lambda$ and $\mu$ are the weakly-increasing vectors, associated to $p$ and $q$, respectively.
\begin{theorem}{(Branden)}
The linear operator $T_!:x^k\to\frac{1}{k!}x^k$ preserves majorization, i.e., $T_!(p) \succ T_!(q),$ whenever $p \succ q$.
\end{theorem}
 \begin{corollary}
  $p(x) \succ q(x) \Rightarrow \lambda_{max} T_!(p) \geq \lambda_{max} T_!(q)$
 \end{corollary}
\end{multicols}
}

\headerbox{Movement of maximum root}{name=MRT,column=2,row=4,span=2,below=MJRZ}{
By Vieta's formulas the roots of $p(x) = \sum_{i=1} a_i x^i$ have sum $\sum_1^n \lambda_i = -\frac{a_n}{a_{n-1}}$.
\begin{theorem} For $p(x) = \sum_{i=1} a_i x^i$, we have
\[\lambda_{max}(p(x)) \geq \frac{ \lambda_{max} (T_!(p))}{\lambda_{max}(L_n(x))} \geq -\frac{a_{n-1}}{n a_n},\]
 where $L_n(x)$ is the $n$-th Laguerre polynomial
\end{theorem}
}
%%%%%%%%%%%%%%%%%%%%%%%%%%%%%%%%%%%%%%%%%%%%%%%%%%%%%%%%%%%%%%%%%%%%%%%%%%%%%%
%%%%%%%%%%%%%%%%%%%%%%%%%%%%%%%%%%%%%%%%%%%%%%%%%%%%%%%%%%%%%%%%%%%%%%%%%%%%%%
\headerbox{References}{name=Ref,column=0,row=5,span=4,below=MRT}{
\noindent 
 
 S. Fisk. {\em Polynomials, roots, and interlacing.} 2008.

 A. Marcus, D. Spielman, and N. Srivastava. {\em Ramanujan graphs and the solution of the Kadison-Singer problem.} Proceedings of ICM, Seoul, 2014.

 J. Borcea, and P. Br\"{a}nden {\em Hyperbolicity preservers and majorization.} 2010.

}


\end{poster}

\end{document}
